\documentclass{sig-alternate-05-2015}
\begin{document}

% ACM templates include ISBN and DOI...
\isbn{N.A.}
\doi{N.A.}

\title{EE480 Assignment 1: 8-bit Signed Saturation Adder}
\subtitle{Implementor's Notes}

\numberofauthors{1}
\author{
Hank Dietz\\
       \affaddr{Department of Electrical and Computer Engineering}\\
       \affaddr{University of Kentucky, Lexington, KY USA}\\
       \email{\texttt{hankd@engr.uky.edu}}
}

\maketitle
\begin{abstract}
This project was a simple combinatorial design problem to
ensure that students were somewhat comfortable with basic use of
Verilog, including the concept of writing an exhaustive testbench.
\end{abstract}


\section{General Approach}

This assignment required construction of a synthesizable 8-bit
signed saturation adder.  It was hinted that this could be
constructed using a conventional (modular) adder followed by a
check for the inputs having the same sign while the output had a
different sign.  When that check is true, I realized that the
ordinary add sign being 1 meant the saturated value should be
127, and if it is 0, the saturation result should be -128.

I built my saturation adder using three modules,
\texttt{satadd8} (as required by the assignment), \texttt{add8},
and \texttt{fa}.  The modular adder (\texttt{add8}) is a simple
ripple-carry implementation built using full adders
(\texttt{fa}).

Rather than clutter the \texttt{testbench} routine with the
computation of the correct (oracle) result, I wrote a separate
module for that, called \texttt{refsatadd8} -- which uses the
same algorithm as the sample code in the assignment.

% insert a break to roughly level columns...
\vfill\pagebreak

\section{Issues}

It was a little confusing how each variable should be declared
and when it should be updated....  The \texttt{\#1} in
\texttt{testbench} is a hack to ensure that each pair of inputs
is processed in sequence with nothing missed.  There are various
other ways to do this, but I didn't want to implement a clock for
testing what is inherently a combinatorial circuit.

Everything was tested and apparently worked correctly the first
time it made it through the WWW-form compiler and simulator.
However, to confirm that the error detection code in
\texttt{testbench} worked, I deliberately introduced an error
(an incorrect carry of \texttt{cout[2]} into
\texttt{fa4}) and observed the output stating 52,744 correct and
12,792 failed. The assignment did not make clear what format the
faulty outputs should be listed in, so I just composed a format
where each line starts with \texttt{Wrong:}.

\end{document}